\documentclass[a4paper,12pt]{scrartcl}

\usepackage[utf8]{inputenc}
\usepackage[T1]{fontenc}
\usepackage[portuguese]{babel}

\usepackage{listings}
\usepackage{graphicx}
\usepackage{float}
\usepackage{hyperref}
\usepackage{lmodern}
\usepackage{enumerate}

\usepackage{xcolor}

\lstset{
	language=C,
	basicstyle=\ttfamily,
	commentstyle=\color{gray}
}

\lstdefinestyle{BashInputStyle}{
	language=bash,
	basicstyle=\small\sffamily,
	%numbers=left,
	%numberstyle=\tiny,
	numbersep=3pt,
	frame=tb,
	columns=fullflexible,
	backgroundcolor=\color{black!5},
	linewidth=0.9\linewidth,
	xleftmargin=0.1\linewidth
}


\begin{document}

\title{MC970/MO644 -- Programação Paralela}
\subtitle{Laboratório 13 -- Multiplicação de Matrizes Distribuida}
\author{Professor: Guido Araújo \\
			Monitor: Hervé Yviquel}
\date{}

\maketitle


\section{Enunciado}

Neste laboratório, o objetivo é paralelizar multiplicação de matrizes (C=A.B) na nuvem usando Apache Spark.

\begin{itemize}
	\item A primeira tarefa é implementar a multiplicação de matrizes da maneira que quiserem mas usando as fonções de paralelização de Spark. A implementação esta totalmente livre: pode ser em Scala ou Python usando um Notebook Jupyter ou não.
	\item A segunda tarefa é criar um cluster Spark no Microsoft Azure e testar a execução na nuvem. Depois da execução, devem analisar a paralelização usando a interface gráfica de profiling do Spark e tentar optimizar ao máximo o tempo de execução. Também, devem modificar o tamanho das matrizes para observar o efeito na execução. A implementação deve ser testada com matrizes de float geradas aleatoriamente e de varias tamanhos para mostrar a escalabilidade da implementação (por exemplo 2000, 8000, e 16000). 
\end{itemize}

Caso tenha alguma dúvida, use o Google Groups - para este trabalho está liberado discutir a solução direta do problema. 

\section{Submissões}

A submissão deve ser \textbf{um arquivo} (em zip) contendo o \textbf{código} (com os arquivos necessários para compilar e testar) e um \textbf{relatório} (em pdf). O relatório deve descrever a implementação, apresentar e analisar os resultados das experimentações na nuvem, e descrever as otimizações que foram aplicadas. Alem disso, o relatório deve conter uma captura de ecrã da interface gráfica de profilamento do Spark mostrando a execução nos núcleos do cluster em paralela.

\end{document}
